\documentclass[10pt]{beamer}

\usetheme[progressbar=frametitle]{metropolis}

\usepackage{booktabs}
\usepackage[scale=2]{ccicons}

\usepackage{pgfplots}
\usepgfplotslibrary{dateplot}

\usepackage{xspace}
\newcommand{\themename}{\textbf{\textsc{Rendezvous}}\xspace}

% my configuration 
\usepackage{ragged2e} % for alignment of paragraph
% my configuration
\usepackage[font=small,skip=0pt]{caption} % spacing between figure and image
% my configuration for indicator  (math)
\usepackage{dsfont}
% my configuration: reduce spacing between text and equation
\expandafter\def\expandafter\normalsize\expandafter{%
    \normalsize
    \setlength\abovedisplayskip{0pt}
    \setlength\belowdisplayskip{0pt}
    \setlength\abovedisplayshortskip{0pt}
    \setlength\belowdisplayshortskip{0pt}
}
\newcommand*{\perm}[2]{{}^{#1}\!P_{#2}}%
\newcommand*{\comb}[2]{{}^{#1}C_{#2}}%

\title{Traffic Schedule on 802.11ax MAC}
%\subtitle{}
\date{\today}
\author{Yang Hang \\ Advisor: K.C. Chen}
\institute{Gratitude Institute of Communication Engineering }
\titlegraphic{\hfill\includegraphics[height=1.5cm]{logo}}

\begin{document}

\maketitle
% frame
\begin{frame}{Table of contents}
	\setbeamertemplate{section in toc}[sections numbered]
    \setbeamertemplate{subsection in toc}[subsections numbered]
    % my configuration
\setcounter{tocdepth}{5}
\setcounter{secnumdepth}{5} 
    \tableofcontents%[hideallsubsections]
\end{frame}

\section{Problem of WiFi before 802.11ax}
% frame
\begin{frame}{Problem of WiFi before 802.11ax}
\alert{Background}
BSS is a WLAN with star topology and AP as the center. 
The \textbf{star topology} means AP works as a \textbf{router} who relay the UL traffic to the backbone and the DL traffic to the STAs. 
AP should transmit at least half of the total traffic, we called it DL traffic. 

\alert{Problem} 
Before 802.11ax, AP is seen as a STA who needs to contend with all the other STAs. It is very unfair for DL traffic and for AP, especially in dense scenario. 

\alert{New in 802.11ax}
The difference in 802.11ax makes it possible to solve this problem. 
The big difference is \textbf{trigger-based} UL, so AP will schedule both the UL and DL traffic. 
Later is some idea about the scheduler.  
\end{frame}

% frame
\begin{frame}{Target}
We focus on dense scenario first, meaning large amount of STAs with heavy traffic loading. 
We want the system should work like:
\begin{itemize}
	\item The system doesn't collapse, i.e., channel is not full of collision. (channel utilization approaching 1)
 	\item UL/DL occupying channel with some reasonable partition.
	\item STAs are fair to share the full loaded channel. 
	\item High priority flows still get some performance guarantee (delay and throughput). 
	\item STAs could save more energy.
\end{itemize} 
\end{frame}

\section{Queueing and Scheduling}
% frame
\begin{frame}{Idea of Scheduler}
The problem will be a flow scheduling. See AP as a router, then UL/DL as two queues. 
Following is the options which may be used as scheduler. 
\begin{enumerate}
\item 
	Priority Queueing
\item Round Robin
\item Fair Queueing\\
\begin{itemize}
	\item
	Bit-by-Bit Round Robin (BBRR)
	\item
	Generalized Processor Sharing Model (GPS)
\end{itemize}
\item 
	Quantum Algorithm (deficit round robin (DRR))
\item
	Token Bucket Filters
\end{enumerate}
\alert{802.11ax Consideration} Resource request procedure is needed for AP to maintain the UL queue. 
\end{frame}


% frame
\begin{frame}{Fair Queueing}
As Internet evolved, we may not want routers handling all traffic with FIFO. 
We may want to share bandwidth among multiple users.
Fundamental mechanism for this purpose is through \textbf{queueing}.
We know that \textbf{Round Robin} is a fair queueing when packets have equal size.
\metroset{block=fill}
\begin{block}{Assumption I}
Packet size is fixed.
\end{block}
This assumption is critical in queueing. However, it is \textbf{not} a reasonable assumption in many situations.

For variant packet sizes, we need a bit more work by generalizing from the \textbf{round robin} idea. 
We will see an example, but before that another assumption is needed.
\begin{block}{Assumption II}
Each subqueue is active (non-empty) all the time.
\end{block} 
\end{frame}

% frame
\begin{frame}{Fair Queueing}
\metroset{block=fill}
\begin{exampleblock}{example}
If \textbf{Assumption II} holds, give 3 subqueues $P,Q,R$. 
The arrival packets are as follow figure.
How are the packets scheduled to achieve fair queueing?
\end{exampleblock}
\begin{figure}
\includegraphics[scale=0.5]{./figure/visual_finishing_time.png}
\end{figure}

Actually, as in round robin, the packets are transmitted in order of increasing cumulative data sent by each subqueue, denoted with $C_p$. 
In the figure above, the order is as $Q1,P1,R1,Q2,Q3,R2,Q4\ldots$

A completely equivalent strategy is to send each packet in nondecreasing order of \textbf{virtual finishing time}, calculated under assumption II.
\end{frame}

% frame
\begin{frame}{bit-by-bit Round Robin (BBRR)}

\end{frame}

% frame
\begin{frame}{Generalized Processor Sharing (GPS)}

\end{frame}

% frame
\begin{frame}{Quantum Algorithm (Deficit Round Robin (DRR))}

\end{frame}

\section{My Design for 802.11ax}
%\begin{frame}[allowframebreaks]{References}
%    \bibliography{demo}
%    \bibliographystyle{abbrv}
%\end{frame}

% frame
\plain{Thanks}

\end{document}
