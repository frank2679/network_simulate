\documentclass[10pt]{beamer}

\usetheme[progressbar=frametitle]{metropolis}

\usepackage{booktabs}
\usepackage[scale=2]{ccicons}

\usepackage{pgfplots}
\usepgfplotslibrary{dateplot}

\usepackage{xspace}
\newcommand{\themename}{\textbf{\textsc{Rendezvous}}\xspace}

% my configuration 
\usepackage{ragged2e} % for alignment of paragraph
% my configuration
\usepackage[font=small,skip=0pt]{caption} % spacing between figure and image
% my configuration for indicator  (math)
\usepackage{dsfont}
% my configuration: reduce spacing between text and equation
\expandafter\def\expandafter\normalsize\expandafter{%
    \normalsize
    \setlength\abovedisplayskip{8pt}
    \setlength\belowdisplayskip{8pt}
    \setlength\abovedisplayshortskip{8pt}
    \setlength\belowdisplayshortskip{8pt}
}
\newcommand*{\perm}[2]{{}^{#1}\!P_{#2}}% pernutation
\newcommand*{\comb}[2]{{}^{#1}C_{#2}}% combination

\title{Traffic Schedule on 802.11ax MAC}
%\subtitle{}
\date{\today}
\author{Yang Hang \\ Advisor: K.C. Chen}
\institute{Gratitude Institute of Communication Engineering }
\titlegraphic{\hfill\includegraphics[height=1.5cm]{logo}}

\begin{document}

\maketitle
% frame
\begin{frame}{Table of contents}
	\setbeamertemplate{section in toc}[sections numbered]
    \setbeamertemplate{subsection in toc}[subsections numbered]
    % my configuration
\setcounter{tocdepth}{5}
\setcounter{secnumdepth}{5} 
    \tableofcontents%[hideallsubsections]
\end{frame}

\section{Problem of WiFi before 802.11ax}
% frame
\begin{frame}{Problem of WiFi before 802.11ax}
\alert{Background}
BSS is a WLAN with star topology and AP as the center. 
The \textbf{star topology} means AP works as a \textbf{router} who relay the UL traffic to the backbone and the DL traffic to the STAs. 
AP should transmit at least half of the total traffic, we called it DL traffic. 

\alert{Problem} 
Before 802.11ax, AP is seen as a STA who needs to contend with all the other STAs. It is very unfair for DL traffic and for AP, especially in dense scenario. 

\alert{New in 802.11ax}
The difference in 802.11ax makes it possible to solve this problem. 
The big difference is \textbf{trigger-based} UL, so AP will schedule both the UL and DL traffic. 
\end{frame}

% frame
\begin{frame}{Target}
We focus on dense scenario first, meaning large amount of STAs with heavy traffic loading. 
We want the system should work like:
\begin{enumerate}
	\item The system doesn't collapse, i.e., channel is not full of collision. (channel utilization approaching 1)
 	\item UL/DL occupying channel with some reasonable partition.
	\item High priority flows still get some performance guarantee (delay and throughput). 
	\item STAs should be energy-efficient.
	\item It's better to give user performance guarantee.
\end{enumerate} 
Since we always measure a network from the systematic aspect, it is really important to measure network from users' view.
\end{frame}

\section{Queueing and Scheduling}
% frame
\begin{frame}{Idea of Scheduler}
The problem will be a flow scheduling. See AP as a router, then UL/DL as two queues. 
Following is the options which may be used as scheduler. 
\begin{enumerate}
\item 
	Priority Queueing
\item Round Robin
\item Fair Queueing\\
\begin{itemize}
	\item
	Bit-by-Bit Round Robin (BBRR)
	\item
	Generalized Processor Sharing Model (GPS)
\end{itemize}
\item 
	Quantum Algorithm (deficit round robin (DRR))
\item
	Token Bucket Filters
\end{enumerate}
\alert{802.11ax Consideration} Resource request procedure is needed for AP to maintain the UL queue. 
\end{frame}

% frame
\begin{frame}{Fair Queueing--Two Assumptions}
As Internet evolved, we may not want routers handling all traffic with FIFO. 
We may want to share bandwidth among multiple users.
Fundamental mechanism for this purpose is through \textbf{queueing}.
We know that \textbf{Round Robin} is a fair queueing when packets have equal size.
\metroset{block=fill}
\begin{block}{Assumption I}
Packet size is fixed.
\end{block}
This assumption is critical in queueing. However, it is \textbf{not} a reasonable assumption in many situations.

For variant packet sizes, we need a bit more work by generalizing from the \textbf{round robin} idea. 
We will see an example, but before that another assumption is needed.
\begin{block}{Assumption II}
Each subqueue is active (non-empty) all the time.
\end{block} 
\end{frame}

% frame
\begin{frame}{Fair Queueing--Virtual Finishing Time}
\metroset{block=fill}
\begin{exampleblock}{Example I}
Give 3 subqueues $P,Q,R$ and only one packet is transmitted at a time.
The arrival packets are as follow figure.
If \textbf{Assumption II} holds, 
how are the packets scheduled to achieve fair queueing?
\end{exampleblock}
\begin{figure}
\includegraphics[scale=0.5]{./figure/visual_finishing_time.png}
\end{figure}

Actually, as in round robin, the packets are transmitted in order of increasing cumulative data sent by each subqueue, denoted with $C_p$. 
In the figure above, the order is as $Q1,P1,R1,Q2,Q3,R2,Q4\ldots$

A completely equivalent strategy is to send each packet in nondecreasing order of \textbf{virtual finishing time}, calculated under assumption II.
\end{frame}

% frame
\begin{frame}{Fair Queueing--Virtual Finishing Time}
The assumption II is also unreasonable for real world. 
If some subqueue is idle (empty) for some time, the \textbf{virtual finishing time} doesn't work well.

\metroset{block=fill}
\begin{exampleblock}{Example II}
All packets are the same size with 1 unit; three subqueues $P,Q,R$. Assume $P$ is idle until $T = 20$, $Q,R$ are busy all the time. At $T=20$, the head packet in $Q, R$ are $Q_{11}, R_{11}$, with virtual finishing times of $T=11$.
For the packet $P_1$, virtual finishing time is $21$. 
\end{exampleblock}
In Example II, the packet $P_1$ should be the transmitted with $Q_{11}, R_{11}$ under the same round. 
However, the virtual finishing order makes $P_1$ much later. 

The trick, as it turns out, is to measure elapsed time not in terms of packet-transmission times (i.e. wallclock time), but rather in terms of rounds of round robin transmission.

The solutions are bit-by-bit Round Robin (BBRR) and generalized processor sharing (GPS).
\end{frame}

% frame
\begin{frame}{bit-by-bit Round Robin (BBRR)}
BBRR is a fair queueing \textbf{without} forwards two assumptions.

"\textbf{Round counter}" $R(t)$ is defined, where $t$ is the time measured in units of transmission time for one bit, $R(t)$ increments by 1 as $t$ increments by $k$ ($k$ active input subqueues); that is, $R(t)$ grows at rate $1/k$.

If a packet of size $S$ starts transmission via BBRR at $R_0 = R(t_0)$, then it will finish when $R(t) = R_0 + S$.

To calculate the virtual BBRR finishing time of a packet $P$, $F_P$, we first record $R_P = R(t_P)$ at the moment of arrival.
\begin{align}
Start &= max(R_P, F_{prev}) \nonumber\\
F_P 	  &= Start + S \qquad (S= \text{ packet size, measued in bits})
\end{align}
\end{frame}

% frame
\begin{frame}{BBRR example}
%\begin{columns}[T,onlytextwidth]
% column
%\column{0.6\textwidth}
\begin{table}
\caption{Given traffic of three subqueues as follows}
\begin{tabular}{llll}
\toprule
Packet & Queue & Size & Arrival time\\
\midrule
$P_1$  & $P$   & 1000 & 0 \\
$P_2$  & $P$   & 1000 & 0 \\
$Q_1$  & $Q$   & 600  & 800 \\
$Q_2$  & $Q$   & 400  & 800 \\
$Q_3$  & $Q$   & 400 & 800 \\
$R_1$  & $R$   & 200 & 1200 \\
$R_2$  & $R$   & 200 & 2100 \\
\bottomrule
\end{tabular}
\end{table}
% column
%\column{0.4\textwidth}
\begin{figure}
\includegraphics[scale=0.3]{./figure/BBRR_eg.png}
\end{figure}
\end{frame}

% frame
\begin{frame}{BBRR example}
For subqueue $P$: \\
$F_P(P_1) = 1000$, 
$F_P(P_2) = 2000$ \\
For subqueue $Q$: \\
$F_P(Q_1) = R(t) + S = 1400$, 
$F_P(Q_2) = 1400+400 = 2200$, 
$F_P(Q_3) = 2200+400 = 2600$\\
For subqueue $R$: \\
$F_P(R_1) = R(1200)+200 = 1200$,
$F_P(R_2) = 1350+200 = 1550$ \\
$R(t)$ is used to measure arrival time with scale $1/k$ ($k$ active subqueues). $F_P$ will not change if other subqueues empty or become active.
% figure
\begin{figure}
\includegraphics[scale=0.28]{./figure/BBRR_cpu.png}
\end{figure}
\end{frame}

% frame 
\begin{frame}{Weighted Fair Queueing}
Though BBRR is a good model for fair queueing, it needs a bit more to design for \textbf{weighted fair queueing} (WFQ).

If we want to implement a WFQ with BBRR, we may need to build multiple subqueue for one flow. 

Another generalized queueing called \textbf{Generalized Processor Sharing} (GPS) is straightforwardly to WFQ. 
\end{frame}

% frame
\begin{frame}{Generalized Processor Sharing (GPS) model}
For the GPS model, assume $N$ input subqueues, and the $i$th subqueue, $0<i<N$, is to receive fraction $\alpha_i>0$, where $\alpha_0+\alpha_1+\ldots+\alpha_{N-1}=1$.
If at some time a set $A$ of input subqueues is active, say $A=\lbrace 0,2,4\rbrace$, then subqueue 0 will receive faction $\frac{\alpha_0}{\alpha_0+\alpha_2+\alpha_4}$, and subqueue 2,4 similarly.
Thus when given different $\alpha_i$, GPS will be a WFQ.

Instead of BBRR round counter $R(t)$, another \textbf{virtual clock} $VC(t)$ is used that runs at an increased rate $1/\alpha\geq 1$ where $\alpha$ is the sum of $\alpha_i$ for the active subqueues. 
Similarly virtual finishing time $F_P$ is given:
\begin{align}
F_P = max(VC(now), F_{prev}) + \frac{S}{r\alpha_i},\ (r \text{ is total output rate of outer})
\end{align}
The same with BBRR, a packet's GPS finishing time does not depend on any later arrivals or idle periods on other subqueues. 
\end{frame}

% frame
\begin{frame}{GPS model}
GPS finishing time: the theoretical finishing time based on \textbf{parallel multi-packet} transmissions under the GPS model. 

Then GPS allows packet to be transmitted immediately when it comes to an idle subqueue. See next example. 
\begin{table}
\caption{Flow Specification $r=1$, each subqueue is allocated an equal share of bandwidth}
\begin{tabular}{lll}
\toprule
$P$ & $Q$ & $R$ \\
\midrule
$P_1: T =0, L=1$ & $Q_1: T=0, L=2$& $R_1:T=10, L=5$\\
$P_2:T=2, L=10$ & $Q_2:T=4, L=6$ & \\
\bottomrule
\end{tabular}
\end{table}
\end{frame}

% frame
\begin{frame}{GPS model--example}
In this case, $VC(t)$ increments with rate $N/k$, $N=3, k:$ active subqueues number. \\
$VC(2) = 2*3/2=3$, 
$VC(4) = 3+2*3/2 = 6$,
$VC(10) = 6+6*3/2 = 15$,
$VC(19) = 15+9 = 6$,
$VC(23) = 24+4*3/2 = 30$ \\
For subqueue $P$:
$F_P(P_1) = 3$, 
$F_P(P_2) = 3+10/(1/3) = 33$\\
For subqueue $Q$:
$F_P(Q_1) = 6$, 
$F_P(Q_2) = 6+6/(1/3) = 24$\\
For subqueue $R$:
$F_P(R_1) = VC(10) + 5/(1/3) = 30$\\
\begin{figure}
\includegraphics[scale=0.23]{./figure/GPS_cpu.png}
\end{figure}
\end{frame}

% frame
\begin{frame}{GPS vs BBRR}
BBRR is a special case of GPS model with $\alpha_i = 1/N$, i.e. equal sharing the bandwidth.
And from above, the scale of $VC(t)$ is $N$ times of $R(t)$.

The GPS has an advantage of generalizing straightforwardly to weighted fair queuing.

However, the implementation of above methods cost $O(logn)$. 
What's more, when number of flow is large amount, it is not easy to maintain so many subqueues.
Actually, there is another way of implementation, called Stochastic Fair Queueing, which is an approximation to fair queueing. It maintains fixed number of subqueues. Here, I won't introduce. \\
Next, I will introduce another method approaching WFQ which costs only $O(1)$. 
\end{frame}

% frame
\begin{frame}{Quantum Algorithm (Deficit Round Robin (DRR))}
DRR gives each subqueue a \textbf{quantum}.
DRR service the subqueues in round robin fashion but each time we take as many packets as we can such that the total size does not exceed the quantum.

Fairness can be improved significantly by keeping track of the difference between the quantum and what was actually sent; this is called \textbf{deficit}.
\metroset{block=fill}
\begin{exampleblock}{Example IV}
The quantum is 1000 bytes and a steady stream of 600 byte packets arrives on one subqueue, they are sent as follows
\end{exampleblock}
\begin{tabular}{llll}
\toprule
round & quantum+prev deficit & packets sent & new deficit\\
\midrule
1 & 1000 & 1 & 400 \\
2 & 1400 & 2,3 & 200 \\
3 & 1200 & 4,5 & 0 \\
\bottomrule
\end{tabular}

If a subqueue is ever empty, its deficit should immediately \textbf{expire}. 
\end{frame}

% frame
\begin{frame}{Token Bucket Filter}
Token bucket filters provide an alternative to fair queueing for providing a traffic allocation to each of several groups. 
The main \textbf{difference} between fair queueing and token bucket is that if one subqueue is idle, fair queueing distributes that sender's bandwidth among the other senders. 
Token bucket does not: the bandwidth a subqueue is allocated is a \textbf{bandwidth cap}.

This forced bandwidth cap is not suitable for wifi. 
That is because the spectrum resource is public for use, we needn't set a bound for some traffic to access the free resource. 
\end{frame}

\section{My Design for 802.11ax}
% frame
\begin{frame}{Single layer WFQ}
I first design a single-layer WFQ satisfying DL/UL scheduling.
The WFQ is implemented with deficit round robin (DRR) algorithm. 
The reason is that, UL/DL is separated in time domain.
Round Robin is also a robust algorithm. 
Then the two key parameter of the algorithm is $D_{dl}, D_{ul}$, representing quantum for DL/UL. 
\begin{figure}
\includegraphics[scale=0.3]{./figure/single_layer.png}
\end{figure}
\end{frame}

% frame
\begin{frame}{Hierarchical WFQ}
Additional function is to support priority traffic, we design a hierarchical queueing as following figure. 
Since GPS model is based on parallel multi-packet transmission, it is like MU PHY. 
So implement sub-WFQ with GPS model.

With the design above, we can give priority traffic some performance guarantee, but not \textbf{user} performance guarantee. 
\begin{figure}
\includegraphics[scale=0.2]{./figure/hierarchical_queueing.png}
\end{figure}
\end{frame}

% frame
\begin{frame}{In progress}
Study the analysis of the model. 
\end{frame}
%\begin{frame}[allowframebreaks]{References}
%    \bibliography{demo}
%    \bibliographystyle{abbrv}
%\end{frame}

% frame
\plain{Thanks}

\end{document}

