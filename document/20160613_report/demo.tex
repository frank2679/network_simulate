\documentclass[10pt]{beamer}

\usetheme[progressbar=frametitle]{metropolis}

\usepackage{booktabs}
\usepackage[scale=2]{ccicons}

\usepackage{pgfplots}
\usepgfplotslibrary{dateplot}

\usepackage{xspace}
\newcommand{\themename}{\textbf{\textsc{Rendezvous}}\xspace}

% my configuration 
\usepackage{ragged2e} % for alignment of paragraph
% my configuration
\usepackage[font=small,skip=0pt]{caption} % spacing between figure and image
% my configuration for indicator  (math)
\usepackage{dsfont}
% my configuration: reduce spacing between text and equation
\expandafter\def\expandafter\normalsize\expandafter{%
    \normalsize
    \setlength\abovedisplayskip{0pt}
    \setlength\belowdisplayskip{0pt}
    \setlength\abovedisplayshortskip{0pt}
    \setlength\belowdisplayshortskip{0pt}
}
\newcommand*{\perm}[2]{{}^{#1}\!P_{#2}}%
\newcommand*{\comb}[2]{{}^{#1}C_{#2}}%

\title{Traffic Schedule on 802.11ax MAC}
%\subtitle{}
\date{\today}
\author{Yang Hang \\ Advisor: K.C. Chen}
\institute{Gratitude Institute of Communication Engineering }
\titlegraphic{\hfill\includegraphics[height=1.5cm]{logo}}

\begin{document}

\maketitle
% frame
\begin{frame}{Table of contents}
	\setbeamertemplate{section in toc}[sections numbered]
    \setbeamertemplate{subsection in toc}[subsections numbered]
    % my configuration
\setcounter{tocdepth}{5}
\setcounter{secnumdepth}{5} 
    \tableofcontents%[hideallsubsections]
\end{frame}

\section{Problem of WiFi before 802.11ax}
% frame
\begin{frame}{Problem of WiFi before 802.11ax}
\alert{Background}
BSS is a WLAN with star topology and AP as the center. 
The \textbf{star topology} means AP works as a \textbf{router} who relay the UL traffic to the backbone and the DL traffic to the STAs. 
AP should transmit at least half of the total traffic, we called it DL traffic. 

\alert{Problem} 
Before 802.11ax, AP is seen as a STA who need to contend with all the other STAs. It is very unfair for DL traffic and for AP, especially in dense scenario. 

\alert{New in 802.11ax}
The difference in 802.11ax makes it possible to solve this problem. 
The big difference is trigger-based UL, which means AP will schedule both the UL and DL traffic. 
Later is some idea about the scheduler.  
\end{frame}

\section{Idea of Scheduler}
% frame
\begin{frame}{Idea of Scheduler}
See the UL/DL as two queues. AP schedules the UL transmission by send a TF first. 
Following is the options which can be used as scheduler. 
\begin{enumerate}
\item (Weighted) Round Robin
\item (Weighted) Fair Queueing\\
 Generalized Processor Sharing Model (GPS)
\end{enumerate}
Resource Request procedure is needed for AP to maintain the UL queue. 
\end{frame}

%\begin{frame}[allowframebreaks]{References}
%    \bibliography{demo}
%    \bibliographystyle{abbrv}
%\end{frame}

% frame    
\plain{Thanks}    
    
\end{document}
